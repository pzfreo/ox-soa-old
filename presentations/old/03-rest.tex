\newif\iffaked \fakedfalse
\iffaked
\documentclass{sepslide-soa-faked} % jim has mucked up xdvi settings
\else
\documentclass{sepslide-soa}
\fi

%\usepackage{array}
%\iffaked\else
%\makeatletter
%\def\@tabular{\leavevmode \hbox \bgroup $\color{TextColor} \col@sep \tabcolsep \let \d@llarbegin \begingroup \let \d@llarend \endgroup \@tabarray} % $
%\makeatother
%\fi

\title{REST}
\topic{3}{REST}[PineGreen]
% \newcolor{TitleColor}{named}{Blue}
\indexfile{00-index.pdf}

\begin{document}

\begin{slide}
  \Title
\end{slide}

\begin{slide}
  \Contents
\end{slide}

\begin{slide}
\Heading{Representational state transfer (REST)}
\begin{itemize}
\item Roy Fielding, a principal author of HTTP
\item PhD thesis \textit{Architectural Styles and the Design of Network-based Software Architectures} (2000)
\begin{quote}
more about evaluation than a cookbook
\end{quote}
\item subsequent article \textit{Principled Design of the Modern Web Architecture} (ACM TOIT 2:2, 2002)
\item Richardson \& Ruby, \textit{RESTful Web Services}
\item architectural patterns of the web
\item taking HTTP seriously as a distributed computing protocol:
fixed few verbs, emphasis on the nouns
\end{itemize}
\end{slide}

\begin{slide}
\Heading{Architectural styles}
\includegraphics[width=\textwidth]{diagrams/rest-constraints}
\end{slide}

\begin{slide}
\Subheading{Client--server (CS)}
\begin{itemize}
\item server offers services, listens for requests
\item client sends request, waits for response
\item transient, triggering client; persistent, reactive server
\item separation of concerns: user interface from behaviour
\item improves \emph{portability} to a new user interface
\item improves \emph{scalability} by simplifying components
\item improves \emph{evolvability} by allowing independent evolution of components
\end{itemize}
\end{slide}

\begin{slide}
\Subheading{Replication (RR) and caching (\$)}
\begin{itemize}
\item \emph{replicated repository}: multiple servers provide same service
\item present the illusion of a single, centralized service
\item improves \emph{performance}: latency, redundancy
\item \emph{maintaining consistency} the primary challenge
\bigskip
\item a variation: \emph{caching} responses for later reuse
\item effectively a replication of a fragment
      (typically, potential data set is huge or infinite)
\item responses explicitly or implicitly labelled cacheable or not
\item lazy or active replication
\item less effective than full replication, but cheaper and simpler
\end{itemize}
\end{slide}

\begin{slide}
\Subheading{Stateless (S)}
\begin{itemize}
\item each request from client must carry all necessary context
\item no \emph{session state} stored on server~--- kept entirely on client
\item (\emph{resource state} is a different matter)
\item improves \emph{visibility} for monitoring
\item improves \emph{reliability} by simplifying recovery from partial failure
\item improves \emph{scalability} by allowing server to free resources quickly
\item improves \emph{evolvability} by simplifying server, cache 
\item decreases \emph{performance} by increasing overhead
\end{itemize}
\end{slide}

\begin{slide}
\Subheading{Layered systems (LS)}
\begin{itemize}
\item hierarchical arrangement
\item layer provides services to layer above, uses services from layer below
\item improves \emph{evolvability} and \emph{reusability} through abstraction
\item decreases \emph{performance} through overhead, latency
\bigskip
\item layered-client-server (LCS) adds proxy and gateway components to CS
\item \emph{proxy} acts as shared server for one or more clients, forwarding (maybe translated) requests
\item \emph{gateway} appears as normal server, but forwards (maybe translated) requests to lower layers: load balancing, security
\end{itemize}
\end{slide}

\begin{slide}
\Subheading{Uniform interface (U)}
\begin{itemize}
\item improves \emph{simplicity} and \emph{visibility}
\item decreases \emph{efficiency} through possible data translations
\bigskip
\item for REST, optimized for large-grain hypermedia data transfer
\item identification of \emph{resources}
\item manipulation of resources through \emph{representations}
\item \emph{self-descriptive} messages
\item hypermedia as the engine of application \emph{state}
\end{itemize}
\end{slide}

\begin{slide}
\Subheading{Virtual machine (VM) and code-on-demand (COD)}
\begin{itemize}
\item mobile code
\item dynamically relocate processing between data source and destination
\item improves \emph{performance} by relocating code near data
\item data element must be transformed into component
\bigskip
\item extend client functionality by downloading applets/scripts
\item virtual machine to provide controlled environment
\item improves \emph{simplicity} and \emph{extensibility} of client
\item reduces \emph{visibility}
\item not a big part of REST-based SOA (yet: cf AJAX)
\end{itemize}
\end{slide}

\begin{slide}
\Heading{The REST architectural style}
\begin{itemize}
\item `uniform, layered, cached-client, stateless-server, with code-on-demand'
\item data elements
\item connectors
\item components
\bigskip
\item some good reading at \url{http://www.prescod.net/rest/}
\end{itemize}
\end{slide}

\begin{slide}
\Subheading{Resources}
\begin{itemize}
\item key idea: every data element is a \emph{resource}
\begin{itemize}
\item named document
\item temporal service (`today's lunch menu at Rewley')
\item collection of other resources
\item non-virtual object (`Marilyn Monroe')
\item anything you might want reference, annotate, or act upon
\item abstractly, just a time-dependent set of values
\end{itemize}
\item resources referenced by URIs
\item generality
\item late binding of reference to representation
\item insulation of reference from representation
\item quality of identifier proportional to effort spent maintaining validity
\end{itemize}
\end{slide}

\begin{slide}
\Subheading{Representations}
\begin{itemize}
\item state of resource captured and transferred between components
\item might be current or desired future state
\item represented as data plus metadata (name--value pairs)
\item think document\maths{+}headers, or HTTP message
\item metadata includes control data, media type
\item one resource might have several representations
\item selected via separate URIs, or via \emph{content negotiation}
\end{itemize}
\end{slide}

\begin{slide}
\Subheading{Interaction}
\begin{itemize}
\item abstract interface for component communication
\item stateless interactions:
\begin{itemize}
\item connectors need not retain application state between requests
\item interactions can be processed in parallel, naively
\item intermediary may view and understand request in isolation
\item reusability of cached response can be determined from response itself
\end{itemize}
\item request parameters: control data, target URI, optional representation
\item response parameters: control data, optional resource metadata, optional representation
\end{itemize}
\end{slide}

\begin{slide}
\Subheading{Connectors}
\begin{itemize}
\item initiating \emph{client}
\item reactive \emph{server}
\item \emph{cache}, either at client or at server, maybe shared
\item \emph{resolver} translates URIs into locations (eg DNS, DOI)
\item \emph{tunnel}, relaying communication across a connection boundary
\end{itemize}
\end{slide}

\begin{slide}
\Subheading{Components}
\begin{itemize}
\item \emph{user agent} uses client connector to initiate response (eg browser)
\item \emph{origin server} uses server connector to govern namespace for requested resource
\item \emph{proxy} and \emph{gateway} act as both client and server
\end{itemize}
\end{slide}

\begin{slide}
\Subheading{State transitions}
\begin{itemize}
\item control state concentrated into resource representations
\item embedded resource identifiers indicate possible next states
\item allows user to directly manipulate state (eg through browser history, context switching)
\item model application is therefore \emph{an engine that moves from state to state by selecting from alternative transitions in current set of representations}
\end{itemize}
\end{slide}

\begin{slide}
\Subheading{REST-based architecture}
\includegraphics[width=\textwidth]{diagrams/rest-architecture}
\end{slide}

\begin{slide}
\Heading{Resource-oriented architecture}
\begin{itemize}
\item after Richardson \& Ruby, \textit{RESTful WS}
\item action identified in HTTP method, not in payload
\item scoping information in URI
\begin{verbatim}
GET reports/open-bugs HTTP/1.1
\end{verbatim}
\item in contrast to RPC-style interaction
\begin{verbatim}
POST /rpc HTTP/1.1
Host: www.upcdatabase.com
\end{verbatim}
\begin{verbatim}
<?xml version="1.0">
<methodCall>
  <methodName>lookupUPC</methodName> ...
</methodCall>
\end{verbatim}
\item \ldots or hybrid
\begin{verbatim}
http://www.flickr.com/services/rest?method=search&tags=cat
\end{verbatim}
\end{itemize}
\end{slide}

\begin{slide}
\Subheading{RESTful operations}
\begin{quote}
\begin{tabular}{@{}lll@{}}
\textit{CRUD verb} & \textit{HTTP method} & \textit{arguments} \\ \hline
create & PUT & fresh representation \\
retrieve & GET & \\
update & PUT & revised representation \\
delete & DELETE &
\end{tabular}
\end{quote}
\end{slide}

\begin{slide}
\Subheading{PUT vs POST}
\begin{itemize}
\item actually, creation by either PUT to new URI or POST to existing URI
\item typically, create a subordinate resource with a POST to its parent
\item use PUT when client chooses URI; use POST when server chooses 
\item successful POST returns code 201 `Created' with \texttt{Location} header
\item (POST also sometimes used for form submission, but this can be non-uniform)
\end{itemize}
\end{slide}

\begin{slide}
\Subheading{Designing a ROA (Richardson \& Ruby)}
\begin{itemize}
\item figure out dataset
\item split dataset into resources
\end{itemize}
and for each resource:
\begin{itemize}
\item name the resource with URI(s)
\item expose (a subset of) the uniform interface
\item design representations accepted from client
\item design representations served to client
\item integrate with other resources, using links
\item consider typical course of events
\item consider exceptional conditions
\end{itemize}
\end{slide}

\begin{slide}
\Subheading{Asynchronous operations}
\begin{itemize}
\item HTTP is synchronous: request--response
\item what about long-running requests? \emph{deferred synchronous} interaction
\medskip
\item client POSTs request (because not idempotent)
\begin{verbatim}
POST /queue HTTP/1.1
Host: jobservice.com
\end{verbatim}
\begin{verbatim}
Please tell me whether 2^43,112,609 - 1 is prime.
\end{verbatim}
\item server queues task, returns code 202 `Accepted' with URI:
\begin{verbatim}
202 Accepted
Location: http://jobservice.com/queue/job11a4f9
\end{verbatim}
\item client polls resource:
\begin{verbatim}
GET /queue/job11a4f9 HTTP/1.1
\end{verbatim}
getting either status report or result
\end{itemize}
\end{slide}

\begin{slide}
\Subheading{URI design}
\begin{itemize}
\item URIs should be meaningful and well-structured
\item client should be able to construct URI to access a resource
  (increases \emph{surface area})
\item use paths to separate elements of hierarchy, general to specific
\item use punctuation to separate items at same hierarchical level
\item commas when order matters (eg coordinates), semicolons otherwise
\item use query variables only for `arguments'
\item URIs denote resources, not operations
  (unless the operation is itself something you might CRUD)
\item some (eg TBL) say URIs should be opaque\ldots why?
  % can stay constant when resource state changes: eg people change name
  % cf weblog entries: original date better than title
\end{itemize}
\end{slide}

\begin{slide}
\Subheading{Representations}
\begin{itemize}
\item may be human-readable, but should be computer-oriented
\item eg collection of values, not picture of graph
  (unless graphing is the service!)
\item need not be XML: plain text, key--value pairs, JSON, \ldots
\item outgoing representations should be acceptable as incoming too: \\
  client may GET, update, PUT
\item use return codes, not outgoing representations, for error conditions
\end{itemize}
\end{slide}

\begin{slide}
\Subheading{Conditional GET}
\begin{itemize}
\item save bandwidth by not resending unchanged representations
\item requires collaboration between client and server
\item server sends \verb"Last-Modified" header with representation
\item client requests \verb"If-Modified-Since" 
\item server may return code 304 `Not Modified', and no representation
\item similarly, \verb"ETag" (hashcode) and \verb"If-None-Match"
\item if client requests both \verb"If-Modified-Since" and \verb"If-None-Match", server should send representation only if representation is modified and has different entity tag
\end{itemize}
\end{slide}

\begin{slide}
\Subheading{The trouble with cookies}
\begin{itemize}
\item opaque data sent by server to client
\item often a key into server-side table of session state
\item violates principles of statelessness (ie, a hack)
  \begin{itemize}
  \item client cannot control their own session state \\
    (breaks the `back' button on the browser)
  \item server must store this state (indefinitely)
  \item much easier for client and server to get out of sync
  \end{itemize}
\end{itemize}
\end{slide}

\begin{slide}
\Subheading{Interfaces for RESTful services}
\begin{itemize}
\item how to describe a RESTful web service?
\item what is to REST as WSDL is to SOAP?
\item \emph{Web Application Description Language} (WADL):
  \begin{itemize}
  \item site map of resources
  \item referential and causal links between resources
  \item methods applicable to each resource
  \item resource representation formats
  \item all machine-processable
  \end{itemize}
\item without something like WADL, is REST service-oriented?
\end{itemize}
\end{slide}

\begin{slide}
\Heading{Example: social bookmarking}
\begin{itemize}
\item user \emph{accounts}
\item \emph{bookmarks}
\item \emph{tags} for bookmarks
\item \emph{bundles} of tags for a user
\medskip
\item each bookmark accumulates \emph{currency}, \emph{popularity}, \emph{tag cloud} \\
  (but we won't model those)
\end{itemize}
\end{slide}

\begin{slide}
\Subheading{Structure}
\begin{quote}
\includegraphics[height=0.8\textheight]{diagrams/delicious.eps}
\end{quote}
\end{slide}

\begin{slide}
\Subheading{Accounts}
\begin{flushleft}
\begin{tabular}{@{}ll@{}}
\textit{intention} & \textit{operation} \\ \hline
create account & \verb"POST /users" \\
view account & \verb"GET /users/<username>" \\
modify account & \verb"PUT /users/<username>" \\
delete account & \verb"DELETE /users/<username>" 
\end{tabular}
\end{flushleft}
\end{slide}

\begin{slide}
\Subheading{Bookmarks}
\begin{flushleft}
\begin{tabular}{@{}ll@{}}
\textit{intention} & \textit{operation} \\ \hline
post bookmark & \verb"POST /users/<username>/bookmarks" \\
fetch bookmark & \verb"GET /users/<username>/bookmarks/<uri>" \\
modify bookmark & \verb"PUT /users/<username>/bookmarks/<uri>" \\
delete bookmark & \verb"DELETE /users/<username>/bookmarks/<uri>" \\
user's last post? & use conditional \verb"GET" \\
user's posting history & \verb"GET /users/<username>/calendar" \\
fetch filtered history & \verb"GET /users/<username>/calendar/<tag>"
\end{tabular}\hspace*{-1in}
\end{flushleft}
\end{slide}

\begin{slide}
\Subheading{Searching}
\begin{flushleft}
\begin{tabular}{@{}ll@{}}
\textit{intention} & \textit{operation} \\ \hline
all user's bookmarks & \verb"GET /users/<username>/bookmarks" \\
\ldots by tag & \verb"GET /users/<username>/bookmarks/<tag>" \\
search & \verb"GET /users/<username>/bookmarks?<query>" 
\end{tabular}
\end{flushleft}
\end{slide}

\begin{slide}
\Subheading{Socialising}
\begin{flushleft}
\begin{tabular}{@{}ll@{}}
\textit{intention} & \textit{operation} \\ \hline
recent bookmarks & \verb"GET /recent" \\
\ldots by tag & \verb"GET /recent/<tag>" \\
who bookmarked uri? & \verb"GET /uris/<uri>" \\
\end{tabular}
\end{flushleft}
\end{slide}

\begin{slide}
\Subheading{Tags and bundles}
\begin{flushleft}
\begin{tabular}{@{}ll@{}}
\textit{intention} & \textit{operation} \\ \hline
user's vocabulary & \verb"GET /users/<username>/tags" \\
rename tag & \verb"PUT /users/<username>/tags/<tag>" \\
user's bundles & \verb"GET /users/<username>/bundles" \\
bundle tags & \verb"POST /users/<username>/bundles" \\
fetch bundle & \verb"GET /users/<username>/bundles/<bundle>" \\
modify bundle & \verb"PUT /users/<username>/bundles/<bundle>" \\
delete bundle & \verb"DELETE /users/<username>/bundles/<bundle>"
\end{tabular}
\end{flushleft}
\end{slide}

\begin{slide}
  \Listofslides
\end{slide}

\begin{slide}
  \Timetable
\end{slide}

\end{document}
