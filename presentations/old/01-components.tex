\newif\iffaked \fakedfalse
\iffaked
\documentclass{sepslide-soa-faked} % jim has mucked up xdvi settings
\else
\documentclass{sepslide-soa}
\fi

\title{Components}
\topic{1}{Software components}[YellowGreen]
% \newcolor{TitleColor}{named}{Blue}
\indexfile{00-index.pdf}

\begin{document}

\begin{slide}
  \Title
\end{slide}

\begin{slide}
  \Contents
\end{slide}

\begin{slide}
\Heading{Mass-produced software components}
\begin{quote}
\ifpdf
\includegraphics[width=\textwidth]{diagrams/MDM-LECTURE.png}%
\else
\includegraphics[width=\textwidth]{diagrams/MDM-LECTURE.eps}%
\fi
\end{quote}
Doug McIlroy, NATO conference on Software Engineering, 1968
\end{slide}

\begin{slide}
\Subheading{Catalogue of components}
\begin{itemize}
\item industrialism vs crofting, engineering vs craft
\item orthogonal families of interchangeable components
\item varying robustness, generality, performance, precision\ldots
\item choice tailored to individual use
\item no significant penalty for choice
\end{itemize}
\end{slide}

\begin{slide}
\Subheading{Component variation}
\begin{itemize}
\item numerical routines for computing sine function
\item varying precision
\item floating or fixed point
\item argument range ($0$--$\pi/2$, $0$--$2\pi$, $-\pi/2$--$\pi/2$, $-\pi$--$\pi$, arbitrary)
\item validation (out of range, loss of significance\ldots)
\item time--space tradeoff by table lookup: table size, quantization
\item expected call sequences (eg successive calls for sine and cosine of same angle)
\end{itemize}
\end{slide}

\begin{slide}
\Subheading{Parameters}
\begin{itemize}
\item generality (which parameters adjustable at run-time)
\item precision (length, width, size)
\item robustness (accuracy versus performance)
\item performance (time vs space)
\item interface (eg for errors: alternate return, error code, exception)
\item algorithm
\item data representation
\end{itemize}
\end{slide}

\begin{slide}
\Subheading{Pragmatics}
\begin{itemize}
\item `mass production' of models, not of replicates
\item can't hand-craft each component: they must be generated
\item (hence generics, partial evaluation, product lines\ldots: 
  systematized parametrization)
\item distinction between component suppliers and consumers
\item chicken and egg problem with a market
\item standardization: neither too early nor too late
\end{itemize}
\end{slide}

\begin{slide}
\Heading{Modularity}
\begin{itemize}
\item structured programming: disciplined control flow
\begin{quote} Dijkstra, \textit{Goto Considered Harmful}, 1968 \end{quote}
\item modules: information hiding
\begin{quote} Parnas, \textit{Criteria for Decomposing Systems}, 1972 \end{quote}
\item data abstraction: smart data structures and dumb code
\begin{quote} Liskov, \textit{Abstract Data Types}, 1974 \end{quote}
\item object orientation: polymorphism, subtyping, inheritance
\begin{quote} Dahl, Myhrhaug, Nygaard, \textit{Simula}, 1967 \end{quote}
\end{itemize}
\end{slide}

\begin{slide}
\Heading{Component-oriented programming}
\begin{itemize}
\item Clemens Szyperski, \textit{Component Software: Beyond OOP}
\item \emph{software components} enable practical reuse of software parts
\item composing components improves quality, supports rapid development,
increases flexibility
\item component software expected to be \emph{the} cornerstone of software
in the years to come
\item \ldots but we've been saying that since 1968
\item \ldots however, recent developments in language technology offer
renewed hope
\end{itemize}
\end{slide}

\begin{slide}
\Subheading{Engineering components} % CS-BOOP 1.1
\begin{itemize}
\item use of components is axiomatic in any mature engineering discipline
\item by analogy, should be in software engineering too
\item sometimes said that software is too \emph{flexible} for components
\begin{quote}
\emph{The programmer, like the poet, works only slightly removed from pure thought-stuff.} (Fred Brooks)
\end{quote}
\item \ldots but this just indicates immaturity in the discipline
\item Szyperski claims \emph{software engineering now entering maturity}
\end{itemize}
\end{slide}

\begin{slide}
\Subheading{Custom-made \textit{vs} standard software} % CS-BOOP 1.2
\begin{itemize}
\item one extreme: project from scratch, using only programming tools and
libraries 
\item optimally adapted, but expensive, technically
suboptimal, usually late, and requires maintenance
\item hence major trend towards other extreme: project outsourced, COTS software bought
% and parametrized until close enough % that's not at the extreme
\item 
  may involve business reorganization,
  cannot confer any competitive advantage,
  slow to adapt to changing needs
\item component software is a `third way': customized assembly of standard
components; compromise between cost efficiency and flexibility
\end{itemize}
\end{slide}

\begin{slide}
\Subheading{Spectrum between make-all and buy-all} % CS-BOOP Fig 1.1
\begin{quote} 
\includegraphics{diagrams/graphs-1} 
\end{quote}
\label{fig:makeall-vs-buyall}
\end{slide}

\begin{slide}
\Subheading{Critical mass} % CS-BOOP 1.3
\begin{itemize}
\item technical superiority of component technology is not enough
\item \emph{critical mass} also required:
  variety and quality, apparent benefit, strong or numerous sources
\item then use of components becomes inevitable, and proprietary solutions
become unsustainable
\item if such a vortex is plausible, \emph{preparedness} for an emerging
component market is essential
\item preparation involves component-friendly (modular) approach, which
has its own advantages anyway
\end{itemize}
\end{slide}

\begin{slide}
\Subheading{The nature of software} % CS-BOOP 1.4
\begin{itemize}
\item software components initially considered analogous to hardware components
  \begin{quote} software IC, software bus \end{quote}
\item also analogy with hi-fi components, 
  mechanical engineering (gears, nuts, bolts), Lego
\item but software is different:
  deliverable is not product, but \emph{blueprint} for product
\item computer is \emph{factory}, instantiating blueprint
\item \textit{by default, repeated instantiation is trivially easy}
\end{itemize}
\end{slide}

\begin{slide}
\Subheading{Deployable entities} % CS-BOOP 1.5
\begin{itemize}
\item as important to distinguish between software and instances as between
plan and building (`the map is not the territory')
\item confusion between classes and objects
\item similar confusions in entity-relationship modelling, CRC cards
\item understandable: in software, both plan and building representable by
same kind of thing (eg object, bits)
\item plans can be parametrized, applied recursively,
scaled, repeatedly instantiated, whereas instances cannot
\item software is a \emph{generic metaproduct}
\item maths a good model of logical structure of software, but not of
engineering and market aspects
\item moral: SE is a blend of physical and logical engineering
\end{itemize}
\end{slide}

%\begin{slide}
%\Subheading{Units of deployment} % CS-BOOP 1.5
%\begin{itemize}
%\item software component is what is actually deployed in a component-based approach 
%\item objects almost never bought, sold or deployed independently
%\item unit of deployment typically more static: class, or framework of classes
%\item objects forming part of `component instance' are instantiated from classes deployed within the component
%\item thus components themselves not normally instantiated; indeed, component may not have OO implementation at all
%\item object technology mostly used to construct monolithic applications, not components
%\end{itemize}
%\end{slide}

\begin{slide}
\Subheading{Lessons learned} % CS-BOOP 1.6
\begin{itemize}
\item non-incestuous component success stories:
  \begin{itemize} \itemsep=0pt \parskip=0pt \topsep=0pt
  \item modern OSs (co-existing applications)
  \item plug-in architectures (eg Firefox, PhotoShop)
  \item mobile phone apps
  \end{itemize}
\item infrastructure providing \emph{rich foundational functionality}
\item components purchased from \emph{independent providers},
  \emph{deployed by clients}, with \emph{direct meaning to client}
\item different roles for component \emph{construction} and \emph{assembly}
\item large enough for reimplementation not to be cost effective
\item multiple components from different sources coexist
\item \ldots but arbitrary combinations may conflict \\
  (composability need only be highly likely, not guaranteed)
\end{itemize}
\end{slide}

\begin{slide}
\Heading{Component terminology}
\begin{itemize}
\item components
\item objects
\item modules
\item whitebox \textit{vs} blackbox
\item interfaces
\item explicit context dependencies
\item component weight
\item standardization and normalization
\end{itemize}
\end{slide}

\begin{slide}
\Subheading{Components} % CS-BOOP 4.1.1
\begin{itemize}
\item \emph{unit of independent deployment}
  \begin{quote}
  encapsulated constituent features; cannot be partially deployed
  \end{quote}
\item \emph{unit of third-party composition}
  \begin{quote}
  self-contained, well specified
  (interaction through well-defined interfaces)
  \end{quote}
\item \emph{no persistent state}
  \begin{quote}
  indistinguishable from a copy 
  (except perhaps for non-operational attributes like serial number)
  \end{quote}
\item eg database server (but not the database itself)
\end{itemize}
\end{slide}

\begin{slide}
\Subheading{Objects} % CS-BOOP 4.1.2
\begin{itemize}
\item unit of instantiation (unique identity: Grandpa's axe)
\item has state, perhaps persistent
\item encapsulates state and behaviour
\item construction plan: \emph{class} (instantiated) or \emph{prototype} (cloned)
\item initialization: \emph{constructor} (static procedure) or \emph{object
factory} (separate object)
\end{itemize}
\end{slide}

\begin{slide}
\Subheading{Components vs objects} % CS-BOOP 4.1.3
\begin{itemize}
\item typically, component consists of one or more classes or prototype
objects, perhaps with immutable objects capturing initial state and other
such resources
\item but component need not contain classes only, or at all
\item might contain global procedures or static variables; might be
implemented in functional or assembly language
\item (references to) objects created inside component may become visible
from outside
\item but revealed objects do not tell whether component is `all OO' inside
\end{itemize}
\end{slide}

\begin{slide}
\Subheading{Modules} % CS-BOOP 4.1.4
\begin{itemize}
\item components close to modules (Modula-2, Ada)
\item Eiffel treats class as `better' module
\item more recent OO languages (Java, Oberon, etc) keep classes and modules
(or packages) separate
\item modules are minimal components
\item but modules have no persistent immutable resources, apart from
hardwired constants, so are not easily reconfigurable
\item modularity is necessary but not sufficient for component orientation
\end{itemize}
\end{slide}

\begin{slide}
\Subheading{Whitebox vs blackbox} % CS-BOOP 4.1.5
\begin{itemize}
\item blackbox abstraction provides only interface signature and specification
\item whitebox abstraction reveals also implementation
\item blackbox and whitebox reuse of implementations, similarly
\item whitebox reuse usually breaks under revision, which retains contract
but changes implementation
\end{itemize}
\end{slide}

\begin{slide}
\Subheading{Interfaces} % CS-BOOP 4.1.6
\begin{itemize}
\item interface defines component's access points
\item these allow component's clients (perhaps components themselves)
access to services provided
\item interface specifies signature and behaviour
\item component usually provides multiple interfaces, \\ 
catering for different client needs
\item navigation between interfaces
\end{itemize}
\vspace*{-0.5in}
\begin{flushright}
\includegraphics[scale=0.75]{diagrams/component}
\end{flushright}
\end{slide}

\begin{slide}
\Subheading{Explicit context dependencies} % CS-BOOP 4.1.7
\begin{itemize}
\item interfaces describe component's \emph{responsibilities}
\item context dependencies describe component's \emph{rights}
\item specify what deploying environment must provide
\item these too should be explicit
\end{itemize}
\end{slide}

\begin{slide}
\Subheading{Component weight} % CS-BOOP 4.1.8
\begin{itemize}
\item component may minimize context dependencies by bundling all required
services
\item defeats purpose of using components
\item dually, may maximize reuse by `outsourcing' all but core
functionality
\item restricts opportunities for use
\begin{quote}
  \emph{maximizing reuse minimizes applicability}
\end{quote}
\end{itemize}
\end{slide}

\begin{slide}
\Subheading{Trade-off of leanness against robustness}
\begin{quote} 
\includegraphics{diagrams/graphs-2} 
\end{quote}
(note similarity with Slide~\ref{fig:makeall-vs-buyall}: 
reuse across components here \textit{vs} reuse of components there)
\end{slide}

\begin{slide}
\Subheading{Standardization and normalization} % CS-BOOP 4.2
\begin{itemize}
\item shift `sweet spot' from robustness towards leanness by improving
degree of standardization and normalization
\item more stable and widely supported aspect is safer dependency
\item (cf business depending on telephone, now and in past)
\item \emph{horizontal standardization}: across domains (eg WWW)
\item \emph{vertical standardization}: within a domain (eg medical radiology)
\item \emph{normalization} to unify competing standards
  (especially parallel vertical standards, eg image formats in medical
  radiology and radio astronomy)
\end{itemize}
\end{slide}

\begin{slide}
\Subheading{Software component: a definition} % CS-BOOP 4.1.5
  \begin{quote}
  \emph{A software component is a unit of composition with contractually
  specified interfaces and explicit context dependencies only. A software
  component can be deployed independently and is subject to composition by
  third parties.}
  \end{quote}
\begin{itemize}
\item an outcome of \textit{Workshop on Component-Oriented Programming} at ECOOP96
\item combines technical and market-oriented aspects
\end{itemize}
\end{slide}

\begin{slide}
\Heading{Services}
\begin{quote}
\emph{It is not the strongest of the species that survive, nor the most intelligent, but the ones most responsive to change.}
(Darwin)
\end{quote}
\begin{itemize}
\item flexibility is vital
\item processes and systems are becoming more and more complex
\item no longer separate systems, but combined 
\item harmonization and central control does not scale
\item we must embrace heterogeneity and decentralization
\end{itemize}
(Arguments from Josuttis, \textit{SOA in Practice}.)
\end{slide}

\begin{slide}
\Subheading{Service-oriented architecture}
\begin{itemize}
\item services
\begin{quote}
self-contained, loosely coupled units of functionality; may participate in multiple processes, and be implemented with various technologies on various platforms
\end{quote}
\item infrastructure
\begin{quote}
the \emph{enterprise service bus}, for easy and flexible interoperation of services
\end{quote}
\item policies
\begin{quote}
to deal with the fact that large distributed systems are heterogeneous, continuously under maintenance, have multiple owners
\end{quote}
\end{itemize}
\end{slide}

\begin{slide}
\Subheading{Large distributed systems}
SOA is specifically for dealing with large distributed systems:
\begin{itemize}
\item legacy-laden
\item heterogeneous
\item long-lived
\item complex
\item under multiple ownership
\item imperfect
\item redundant (denormalized)
\item intolerant of bottlenecks
\end{itemize}
\end{slide}

\begin{slide}
\Subheading{Postmodern software engineering}
\begin{quote}
  Lots of software is already written and can solve your problems, if
  you can only get it to do what you need. \emph{Postmodern programming} is
  about using code that already exists, and not writing much code
  yourself. This is usually in the form of gluing together or
  configuring other people's code. This is the reality of `enterprise'
  software today. Postmodern programmers recognise this fact and work
  with it rather than pretend that it isn't the case and come up with,
  for example, a programming language or paradigm that would solve all
  programming problems once and for all if only everybody does
  everything properly---that way---the one true way. 
  \textbf{There is no One True Way.}
\end{quote}
\url{http://www.elvis.ac.nz/brain?PostmodernProgramming}
\end{slide}

\begin{slide}
\Subsubheading{PoMoPro principles}
\begin{itemize}
\item examples over documentation
\item source code over binary components
\item loosely structured data over highly structured data
\item dynamic typing over static typing
\item focused components over frameworks
\item composing components over modifying existing applications
\item rich component library over programming tools
\item actual capabilities over intended use (POSIWID)
\item simplify the problem over functional areas 
\end{itemize}
(Nat Pryce, Ivan Moore, \textit{Scrapheap Challenge}, OOPSLA2005)
\end{slide}

\begin{slide}
\begin{minipage}{0.5\textwidth}\raggedright
\vbox to 3.1in{
\Subsubheading{Modernist perspective}
\begin{itemize}
\item absolute truth, right and wrong answers, one big story
\item correctness is everything
\item specification, stepwise refinement, proof
\end{itemize}\vfil}
\end{minipage}
\hfil
\begin{minipage}{0.4\textwidth}\begin{flushright}%
\includegraphics[height=3.1in]{diagrams/115564-004-E6960AAA_royalsoc}%
\end{flushright}\end{minipage}%
\end{slide}

\begin{slide}
\begin{minipage}{0.5\textwidth}\raggedright
\vbox to 3.1in{
\Subsubheading{Postmodern perspective}
\begin{itemize}
\item `incredulity towards meta-narratives'
\item no one privileged view
\item toleration of conflict and contradiction
\item it's all relative
\item `good enough software', `worse is better'
\end{itemize}\vfil}
\end{minipage}
\hfil
\begin{minipage}{0.4\textwidth}\begin{flushright}%
\includegraphics[height=3.1in]{diagrams/Braque-woman-400pix}%
\end{flushright}\end{minipage}%
\end{slide}

\begin{slide}
\Subheading{Some more definitions of services}
\begin{quote}
\emph{loosely coupled software components that interact with each other dynamically via standard Internet technologies}
(Gartner)
\end{quote}
\begin{quote}
\emph{a piece of business logic accessible via the Internet using open standards}
(Microsoft)
\end{quote}
\begin{quote}
\emph{encapsulated, loosely coupled, contracted software functions, offered via standard protocols over the Web}
(DestiCorp)
\end{quote}
\begin{quote}
\emph{a software application identified by a URI, whose interfaces and binding are capable of being defined, described and discovered by XML artifacts and supports direct interactions with other software applications using XML-based messages via Internet-based protocols}
(W3C)
\end{quote}
\end{slide}

\begin{slide}
\Subheading{Observations}
\begin{itemize}
\item services are ideally \emph{stateless}
\item \emph{reuse} has been subordinated to modularity
\item \emph{interfaces} are the crux 
\begin{quote} (a Gartner analyst said that `SOA would be better named \emph{interface-oriented architecture}') \end{quote}
\item signature $\to$ behaviour $\to$ service level
\end{itemize}
\end{slide}

\begin{slide}
\Subheading{Service principles}
\begin{itemize}
\item reusable
\item contracted
\item loosely coupled
\item abstract
\item composable
\item autonomous
\item stateless
\item discoverable
\end{itemize}
(After Erl, \textit{SOA}.)
\end{slide}

\begin{slide}
\Subheading{SOA vs Web Services}
\begin{itemize}
\item WS $=$ WSDL $+$ UDDI $+$ SOAP (roughly speaking)
\item service description, discovery, invocation
\item using standard XML and Internet technologies
\item just one approach to realizing the technical aspects of SOA
\item SOA is not a technology!
\bigskip
\item is service-orientation the same as component-orientation?
\end{itemize}
\end{slide}

\begin{slide}
  \Listofslides
\end{slide}

\begin{slide}
  \Timetable
\end{slide}

\end{document}

%%% Local Variables: 
%%% mode: latex
%%% TeX-master: t
%%% End: 
